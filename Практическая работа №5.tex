\documentclass[a4paper,12pt]{article}
\usepackage[T2A]{fontenc}             
\usepackage{mathtext}                 
\usepackage[utf8]{inputenc}           
\usepackage[english, russian]{babel}   
\usepackage[left=1cm,right=1cm,top=1cm,bottom=2cm]{geometry}
\usepackage{graphicx}
\usepackage{multirow}
\usepackage{makecell}
\author{Комаров А.В.}
\title{Практическая работа №5}

\begin{document}
\maketitle
Дана функция (ДНФ):
$$F=\bar{B}C\bar{D}+\bar{AC}+BD\bar{A}$$

\begin{table}[h]
\renewcommand{\tabcolsep}{0.15cm}
\caption{\label{TRT1} Таблица истинности.}

\begin{tabular}{|c|c|c|c|c|c|c|c|c|c|c|c|c|c|c|c|}
\hline
\multicolumn{4}{|c|}{Аргументы} & \multicolumn{11}{|c|}{Логические операции} \\
\hline
\rule{0cm}{0.5cm}
A & B & C & D & 
\small $\bar{B}$ & 
\small $\bar{B}C$ & 
\small $\bar{D}$&
\small $\bar{B}C\bar{D}$ & 
\small $\bar{A}$ &
\small $\bar{C}$ & 
\small $\bar{A}\bar{C}$ & 
\small $\bar{B}C\bar{D}+\bar{A}\bar{C}$ &
\small $BD$ &
\small $BD\bar{A}$ &
\small $\bar{B}C\bar{D}+\bar{A}\bar{C}+BD\bar{A}$\\
\hline
  0 & 0 & 0 & 0 & 1 & 0 & 1 & 0 & 1 & 1 & 1 & 1 & 0 & 0 & 1 \\
  0 & 0 & 0 & 1 & 1 & 0 & 0 & 0 & 1 & 1 & 1 & 1 & 0 & 0 & 1 \\
  0 & 0 & 1 & 0 & 1 & 1 & 1 & 1 & 1 & 0 & 0 & 1 & 0 & 0 & 1 \\
  0 & 0 & 1 & 1 & 1 & 1 & 0 & 0 & 1 & 0 & 0 & 0 & 0 & 0 & 0 \\
  0 & 1 & 0 & 0 & 0 & 0 & 1 & 0 & 1 & 1 & 1 & 1 & 0 & 0 & 1 \\
  0 & 1 & 0 & 1 & 0 & 0 & 0 & 0 & 1 & 1 & 1 & 1 & 1 & 1 & 1 \\
  0 & 1 & 1 & 0 & 0 & 0 & 1 & 0 & 1 & 0 & 0 & 0 & 0 & 0 & 0 \\
  0 & 1 & 1 & 1 & 0 & 0 & 0 & 0 & 1 & 0 & 0 & 0 & 1 & 1 & 1 \\
  1 & 0 & 0 & 0 & 1 & 0 & 1 & 0 & 0 & 1 & 0 & 0 & 0 & 0 & 0 \\
  1 & 0 & 0 & 1 & 1 & 0 & 0 & 0 & 0 & 1 & 0 & 0 & 0 & 0 & 0 \\
  1 & 0 & 1 & 0 & 1 & 1 & 1 & 1 & 0 & 0 & 0 & 1 & 0 & 0 & 1 \\
  1 & 0 & 1 & 1 & 1 & 1 & 0 & 0 & 0 & 0 & 0 & 0 & 0 & 0 & 0 \\
  1 & 1 & 0 & 0 & 0 & 0 & 1 & 0 & 0 & 1 & 0 & 0 & 0 & 0 & 0 \\
  1 & 1 & 0 & 1 & 0 & 0 & 0 & 0 & 0 & 1 & 0 & 0 & 1 & 0 & 0 \\
  1 & 1 & 1 & 0 & 0 & 0 & 1 & 0 & 0 & 0 & 0 & 0 & 0 & 0 & 0 \\
  1 & 1 & 1 & 1 & 0 & 0 & 0 & 0 & 0 & 0 & 0 & 0 & 1 & 0 & 0 \\
  \hline
\end{tabular}
\end{table} 
Совершенная дизъюнктивная нормальная форма (СДНФ)
$$ F=\bar{A} \land \bar{B} \land \bar{C} \land \bar{D}\lor
\bar{A} \land \bar{B} \land \bar{C} \land D \lor
\bar{A} \land \bar{B} \land C \land \bar{D}\lor
\bar{A} \land B \land \bar{C} \land \bar{D}\lor
\bar{A} \land B \land \bar{C} \land D\lor
\bar{A} \land B \land C \land D\lor
A \land \bar{B} \land C \land \bar{D}$$\\


Совершенная конъюнктивная нормальная форма (СКНФ)
$$ F=(A \land B \land \bar{C} \land \bar{D})\lor(A \land \bar{B} \land \bar{C} \land D)\lor(\bar{A} \land B \land D \land C)\lor(\bar{A} \land B \land C \land \bar{D})\lor(\bar{A} \land B \land \bar{C} \land \bar{D})\lor(\bar{A} \land \bar{B} \land C \land D)\lor(\bar{A} \land \bar{B} \land C \land \bar{D})\lor$$
$$(\bar{A} \land \bar{B} \land \bar{C} \land D)\lor(\bar{A} \land \bar{B} \land \bar{C} \land \bar{D}) $$\\
\begin{table}
\caption{\label{TRT2} Карта карно}
\begin{tabular}{|c|c|c|c|c|}
\hline
\diaghead{\theadfont 1111111111}
{AB}{CD}
& 00 & 01 & 11 & 10 \\
\hline
00 & 1 & 1 & 0 & 1 \\
\hline
01 & 1 & 1 & 1 & 0 \\
\hline
11 & 0 & 0 & 0 & 0 \\
\hline
10 & 0 & 0 & 0 & 1 \\
\hline
\end{tabular}
\end{table} 

\end{document}